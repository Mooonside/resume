% !TEX program = xelatex

\documentclass{resume}
%\usepackage{zh_CN-Adobefonts_external} % Simplified Chinese Support using external fonts (./fonts/zh_CN-Adobe/)
%\usepackage{zh_CN-Adobefonts_internal} % Simplified Chinese Support using system fonts

\begin{document}
\pagenumbering{gobble} % suppress displaying page number

\name{Chen Yifeng}
\basicInfo{
	\email{yifengchen@zju.edu.cn} \textperiodcentered\ 
	\phone{(+86) 138-1836-9294} \textperiodcentered\ 人定义的 英文
	%\linkedin[yifeng19]{https://www.linkedin.com/in/yifeng19/} \textperiodcentered\ 
	\github[Mooonside]{https://www.github.com/Mooonside/}
}
\section{\faGraduationCap\ Education}
\datedsubsection{\textbf{Zhejiang University}, Hangzhou, China}{2018 -- Present}
\textit{M.S.} in  Computer Science \& Technology, expected March 2021\\
\textit{Instructed by}\ \href{https://person.zju.edu.cn/xilics}{Prof.Xi Li}
\datedsubsection{\textbf{Shanghai Jiao Tong University (SJTU)}, Shanghai, China}{2014 -- 2018}
\textit{B.S.} in Computer Science \& Technology\\
\textit{Ranking}\ 31 / 150

\section{\faUsers\ Experience}
\datedsubsection{\textbf{Bidirectional Aggregation Network for Panoptic Segmentation}}{Jun.~2019 -- Nov.~2019}
\role{} {In review of CVPR 2020}
\begin{itemize}
	\item Motivation: Panoptic segmentation includes semantic segmentation and instance segmentation. To alleviate the complementarity between these two tasks, we build a bidirectional connection between them. % Meanwhile, we propose a learning-free algorithm to handle the occlusion between instances.
	\item Method: We design feature aggregation modules to enable information flow between these two tasks. Specifically, we propose a differentiable operator, RoIInlay. It can recover the lost spatial information of instances such that the recovered features can be aggregated to help semantic segmentation. 
	\item Result: Our model based on ResNet-50 achieved state-of-the-art~(43.0\%) w.r.t PQ~(Panoptic Quality). 
\end{itemize}

\datedsubsection{\textbf{Reduce Dependency on Context for Semantic Segmentation}}{Mar.~2019 -- May.~2019}
\begin{itemize}
	\item Motivation: Recent semantic segmentation models rely heavily on context, which may hurt generalizability. To alleviate this, we design a method to reduce the dependency on context for better generalizability.
	\item Method: For an image, we can erase some regions of it by predefined rules. For the untouched regions, we optimize our model to not only predict semantic classes correctly, but also extract consistent features.
	\item Result: We test our model on three sets: the val set, the style-transferred val set, and an unseen dataset. While maintaining comparable results on the val set~(-0.2\%), our model outperforms the baseline by 1.5\% on the style-transferred images and 2\% for the unseen dataset w.r.t. mIoU.
\end{itemize}

\datedsubsection{\textbf{Chat Assistant}}{Mar.5~2017 -- May.7~2017}
\role{SJTU Hackathon 2017} {Development}
\begin{itemize}
	\item Motivation: When chatting with our friends on a mobile phone, we could come across some new terms. To avoid the annoying switch between searching and chatting, we design the Chatting Assistant. 
	\item Method: Based on the Watson API~(by IBM), Chatting Assistant can automatically extract terms of interest, search them online and provide brief introductions for users.
\end{itemize}



\section{\faCogs\ Skills}
\begin{itemize}[parsep=0.5ex]
  \item Programming Languages: Familiar with Python, CUDA, and C++
  \item Deep Learning Platform: Familiar with PyTorch and MXNet
\end{itemize}

\section{\faHeartO\ Honors and Awards}
\datedline{\textit{B-Class Scholarship~(top 10\% students)}, SJTU}{2015年}
\datedline{\textit{IBM Innovation Award}, SJTU Hackathon}{2017年}

\section{\faInfo\ Miscellaneous}
\begin{itemize}
  \item Languages: English - Fluent~(CET-6: 577)
\end{itemize}

%% Reference
%\newpage
%\bibliographystyle{IEEETran}
%\bibliography{mycite}
\end{document}
