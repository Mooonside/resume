% !TEX program = xelatex

\documentclass{resume}
%\usepackage{zh_CN-Adobefonts_external} % Simplified Chinese Support using external fonts (./fonts/zh_CN-Adobe/)
%\usepackage{zh_CN-Adobefonts_internal} % Simplified Chinese Support using system fonts

\begin{document}
\pagenumbering{gobble} % suppress displaying page number

\name{Yifeng Chen}
\basicInfo{
	\email{yifengovo@gmail.com} \textperiodcentered\ 
	\phone{(+86) 138-1836-9294} \textperiodcentered\ 人定义的 英文
	%\linkedin[yifeng19]{https://www.linkedin.com/in/yifeng19/} \textperiodcentered\ 
	\github[Mooonside]{https://www.github.com/Mooonside/}
}
\section{\faGraduationCap\ Education}
\datedsubsection{\textbf{Zhejiang University}, Hangzhou, China}{2018 -- Present}
\textit{M.E.} in  Computer Science \& Technology, expected March 2021\\
\textit{Instructed by}\ \href{https://person.zju.edu.cn/xilics}{Prof.Xi Li}
\datedsubsection{\textbf{Shanghai Jiao Tong University}, Shanghai, China}{2014 -- 2018}
\textit{B.E.} in Computer Science \& Technology
%\textit{Ranking}\ 31 / 150
\vspace{-0.25em}

\section{\faUsers\ Experience}
\datedsubsection{\textbf{Disentangle Positions \& Semantics for Segmentation}} {2020/05 -- Now}
\vspace{-0.3em}
\role{Internship} {Tencent YouTu Lab}
\vspace{-0.5em}
\begin{itemize}
\item Through decoupling positions and semantics, we see a significant improvement of 1.2\% w.r.t. mIoU at 90\% less cost on COCO compared with Non-Local. This work will be submitted to CVPR 2021.
\end{itemize}

\datedsubsection{\textbf{TextRay: Arbitrary-shaped Scene Text Detection}} {2020/01 -- 2020/04}
\vspace{-0.3em}
\role{Research} {Accepted by ACM MM 2020; Oral}
\vspace{-0.5em}
\begin{itemize}
	\item We propose a contour-based method to model texts of arbitrary shapes and design Central Weighting \& Content Loss to facilitate training. 
	\item Our model is 5.5\% higher than the baseline w.r.t. F1-score and achieves S.O.T.A. on CTW1500 dataset.
\end{itemize}


\datedsubsection{\textbf{Bidirectional Aggregation Network for Panoptic Segmentation}} {2019/06 -- 2019/11} 
\vspace{-0.3em}
\role{Research} {Accepted by CVPR 2020; Oral}
\vspace{-0.5em}
\begin{itemize}
	\item We build a bidirectional aggregation path between semantic and instance segmentation to tap their complementarity. This leads to an increment of 1.2\% w.r.t. PQ.
	\item We design RoIInlay to recover the layout of instances. It is faster~(x3) and more accurate~(+1.0\% SQ).
	\item We propose a texture-based algorithm to handle occlusion and observe a 2.0\% improvement on RQ. 
\end{itemize}

\datedsubsection{\textbf{Context Dependency Reduction for Semantic Segmentation}}{2019/03 -- 2019/05} 
\vspace{-0.3em}
\role{Research} {}
\vspace{-0.5em}
\begin{itemize}
		\item We propose a method to reduce the dependency on context for better generalizability. We erase parts of an image and ask the model to predict correct classes and extract consistent features within untouched regions.
	\item We train the model on CityScapes. While maintaining comparable results on the \textit{val} set~(-0.2\%), our model outperforms the baseline by 1.5\% on the style-transferred \textit{val} set and 2\% for the unseen dataset w.r.t. mIoU.
\end{itemize}
\vspace{-1.0em}	
%\datedsubsection{\textbf{Chat Assistant}}{2017/05/05 -- 2017/05/07} 
%\vspace{-0.3em}
%\role{Competition} {SJTU Hackathon 2017}
%\vspace{-0.5em}
%\begin{itemize}
%	\item Motivation: When chatting with our friends on a mobile phone, we could come across some new terms. To avoid the annoying switch between searching and chatting, we design the Chatting Assistant. 
%	\item Method: Based on the Watson API, Chatting Assistant can automatically extract terms of interest, search them online and provide brief introductions for users. Our project won the IBM Innovation Award.
%\end{itemize}

\section{\faFileTextO\ Publication}
	\begin{itemize}
	\item BANet: Bidirectional Aggregation Network with Occlusion Handling for Panoptic Segmentation \\
	IEEE Conference on Computer Vision and Pattern Recognition 2020~(\textbf{CVPR 2020}; \textbf{Oral}) \\
	\textbf{Yifeng Chen}, Guangchen Lin, Xi Li, et al. 
	\item TextRay: Contour-based Geometric Modeling for Arbitrary-shaped Scene Text Detection \\
	ACM International Conference on Multimedia 2020~(\textbf{ACM MM 2020; \textbf{Oral}})\\
	Fangfang Wang, \textbf{Yifeng Chen}, Xi Li, et al. 
\end{itemize}
\vspace{-0.5em}	

\section{\faCogs\ Skills}
\begin{itemize}[parsep=0.5ex]
  \item I am familiar with CUDA programming.
%  \item Deep Learning Platform: Familiar with PyTorch and MXNet
\end{itemize}
\vspace{-0.5em}
%
\section{\faHeartO\ Honors and Awards}
\datedline{\textit{B-Class Scholarship~(top 10\% students)}, SJTU}{2015年}
\datedline{\textit{IBM Innovation Award}, SJTU Hackathon}{2017年}

%\section{\faInfo\ Miscellaneous}
%\begin{itemize}
%  \item Languages: English - Fluent~(CET-6: 577)
%\end{itemize}

%% Reference
%\newpage
%\bibliographystyle{IEEETran}
%\bibliography{mycite}
\end{document}
