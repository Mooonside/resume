% !TeX spellcheck = en_US
% !TEX TS-program = xelatex
% !TEX encoding = UTF-8 Unicode
% !Mode:: "TeX:UTF-8"

\documentclass{resume}
\usepackage{zh_CN-Adobefonts_external} % Simplified Chinese Support using external fonts (./fonts/zh_CN-Adobe/)
% \usepackage{NotoSansSC_external}
% \usepackage{NotoSerifCJKsc_external}
% \usepackage{zh_CN-Adobefonts_internal} % Simplified Chinese Support using system fonts
\usepackage{linespacing_fix} % disable extra space before next section
\usepackage{cite}
\usepackage{amsmath}

\begin{document}
	\pagenumbering{gobble} % suppress displaying page number
	
	\name{陈怡峰}
	
	\basicInfo{
		\email{yifengovo@gmail.com} \textperiodcentered\ 
		\phone{(+86) 13818369294} 
		%\textperiodcentered\ 
%		\linkedin[yifeng]{https://www.linkedin.com/in/yifeng19/} \textperiodcentered\ 
		%\github[Mooonside]{https://www.github.com/Mooonside/}
	}
	
	\section{\faGraduationCap\  教育背景}
	\datedsubsection{\textbf{浙江大学},杭州}{2018 -- 至今}
	\textit{在读硕士研究生}\ 计算机科学与技术\ 预计 2021 年 3 月毕业\\
	\textit{导师}\ \href{https://person.zju.edu.cn/xilics}{李玺} 
	\vspace{-0.25em}
	\datedsubsection{\textbf{上海交通大学},上海}{2014 -- 2018}
	\textit{学士}\ 计算机科学与技术
	\vspace{-0.5em}
	
	\section{\faUsers\ 实习/项目经历}
	\datedsubsection{\textbf{基于动态位置编码的全景分割模型}} {2020/05 -- 2020/11}
	\role{实习} {腾讯-优图实验室}
	在一阶段全景分割框架中引入动态位置编码机制以适应不同任务、场景对位置信息的需求。
	\vspace{-0.2em}
	\begin{itemize}
		%\small
		\item 与SOTA方法相比,我们的模型速度相近~(11.8 FPS),PQ提升了2.1\%至42.3\%,接近两阶段性能。
		\item 动态位置信息编码机制通过IDFT参数化学习全局空间位置相关性,使其能对不同场景、任务动态地编码位置信息。在语义和实例分割分支中引入这一机制,提升了2.7\% $\text{PQ}^\text{St}$和1.5\% $\text{PQ}^\text{Th}$。
		\item 以第一作者身份投稿至CVPR-2021。
	\end{itemize}

	
	\datedsubsection{\textbf{Text Ray:任意形状的文本检测模型}}{2020/01 -- 2020/04}
	\role{研究} {ACM MM 2020 Oral接收}
	对文本检测提出了一种基于轮廓的建模方式,并提出Central Weighting和Content Loss来帮助学习。
	\vspace{-0.2em}
	\begin{itemize}
		\item 把文本的轮廓简化为一个中心点和一族等间隔的射线,并通过多项式拟合来参数化学习射线族。
		\item 提出Central Weighting来挑选高质量的正样本;提出Content Loss在优化参数时考虑其内部关系。
		\item 以F1-score衡量,方法在CTW1500上较baseline提升了5.5\%,达到了SOTA.(78.92\%)。
	\end{itemize}
	\vspace{-0.25em}
	
	\datedsubsection{\textbf{带有双向连接和遮挡处理的全景分割模型}}{2019/06 -- 2019/11}
	\role{研究} {CVPR 2020 Oral接收}
	在全景分割的子任务间建立双向特征连接来利用其互补信息,并提出一个算法来处理遮挡问题。
	\vspace{-0.2em}
	\begin{itemize}
		\item 在语义与实例分割间建立双向特征连接,并设计了SIM和OCM模块, 提升了1.2\% PQ至43.0\%。
		\item 提出RoIInlay算子以恢复实例的空间信息。它相较其他算子速度快~(x3.6),效果更好~(+1.0\% SQ)。
		\item 提出遮挡处理算法,通过比较遮挡区域与候选目标在外观上的相似度来推理遮挡, 提升了2\% RQ。
		% \item 以PQ衡量,模型(基于ResNet-50)在COCO数据集上较baseline提升了约2\%,达到了S.O.T.A.~(43.0\%).
	\end{itemize}
	\vspace{-0.25em}
	
	\datedsubsection{\textbf{降低上下文耦合的语义分割模型}}{2019/03 -- 2019/05}
	\role{研究} {}
	针对模型在上下文改变时性能差的问题提出了一种降低特征与上下文耦合的方法。
	\vspace{-0.2em}
	\begin{itemize}
		\item 通过擦除构造样本对,并在丢失上下文时要求模型对未擦除区域正确地预测类别且提取特征一致。
		\item 以mIoU衡量,模型能在原测试集上维持精度,且在未见过的Apollo数据集上提升约2\%。
	\end{itemize}
	\vspace{-0.5em}
	
	\section{\faFileTextO\ 论文发表}
	\begin{itemize}
		\small
		\item BANet: Bidirectional Aggregation Network with Occlusion Handling for Panoptic Segmentation \\
		IEEE Conference on Computer Vision and Pattern Recognition 2020~(\textbf{CVPR 2020}; \textbf{Oral}) \\
		\textbf{Yifeng Chen}, Guangchen Lin, Xi Li, et al. 
		\item TextRay: Contour-based Geometric Modeling for Arbitrary-shaped Scene Text Detection \\
		ACM International Conference on Multimedia 2020~(\textbf{ACM MM 2020}; \textbf{Oral})\\
		Fangfang Wang, \textbf{Yifeng Chen}, Xi Li, et al. 
	\end{itemize}
	\vspace{-0.5em}	

%	\section{\faCogs\ IT 技能}
%	 % increase linespacing [parsep=0.5ex]
%	\begin{itemize}[parsep=0.5ex]
%		\item 熟悉CUDA编程,有编写测试底层操作的工程经验。
%	%	\item 深度学习框架: 熟悉PyTorch,MXNet,了解TensorFlow
%	\end{itemize}
%	\vspace{-0.5em}
	
	\section{\faHeartO\ 获奖情况}
	\datedline{\textit{国家奖学金}, 浙江大学}{2019-2020学年}
	\datedline{\textit{IBM青年创客奖},上海交通大学Hackathon}{2017年}
	%\datedline{\textit{电院B级奖学金~(授予前10\%)},上海交通大学}{2015年}

	
%	\section{\faInfo\ 其他}
%	% increase linespacing [parsep=0.5ex]
%	\begin{itemize}[parsep=0.5ex]
%		% \item 技术博客: https://mooonside.github.io/
%		\item 语言: 英语 - 熟练(CET-6 577)
%	\end{itemize}
	
	%% Reference
	%\newpage
	%\bibliographystyle{IEEETran}
	%\bibliography{mycite}
\end{document}
