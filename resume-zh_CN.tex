% !TEX TS-program = xelatex
% !TEX encoding = UTF-8 Unicode
% !Mode:: "TeX:UTF-8"

\documentclass{resume}
\usepackage{zh_CN-Adobefonts_external} % Simplified Chinese Support using external fonts (./fonts/zh_CN-Adobe/)
% \usepackage{NotoSansSC_external}
% \usepackage{NotoSerifCJKsc_external}
% \usepackage{zh_CN-Adobefonts_internal} % Simplified Chinese Support using system fonts
\usepackage{linespacing_fix} % disable extra space before next section
\usepackage{cite}

\begin{document}
\pagenumbering{gobble} % suppress displaying page number

\name{陈怡峰}

\basicInfo{
  \email{yifengchen@zju.edu.cn} \textperiodcentered\ 
  \phone{(+86) 138-1836-9294} \textperiodcentered\ 
  \linkedin[yifeng19]{https://www.linkedin.com/in/yifeng19/} \textperiodcentered\ 
  \github[Mooonside]{https://www.github.com/Mooonside/}
}
 
\section{\faGraduationCap\  教育背景}
\datedsubsection{\textbf{浙江大学}, 杭州, 浙江}{2018 -- 至今}
\textit{在读硕士研究生}\ 计算机科学与技术, 预计 2020 年 3 月毕业
\datedsubsection{\textbf{上海交通大学}, 上海}{2014 -- 2018}
\textit{学士}\ 计算机科学与技术

%\section{\faUsers\ 实习/项目经历}
\section{\faUsers\ 项目经历}
\datedsubsection{\textbf{降低上下文耦合的语义分割模型}}{2019年3月 -- 2019年6月}
%\role{实习}{经理: 高富帅}
%xxx后端开发
%\begin{itemize}
%  \item 实现了 xxx 特性
%  \item 后台资源占用率减少8\%
%  \item xxx
%\end{itemize}
\datedsubsection{\textbf{带有双向连接的全景分割模型}}{2019年6月 -- 2019年11月}


\section{\faCogs\ IT 技能}
% increase linespacing [parsep=0.5ex]
\begin{itemize}[parsep=0.5ex]
  \item 编程语言: Python > C++ = CUDA > C
  \item 深度学习框架: PyTorch = MXNet > TensorFlow
\end{itemize}

\section{\faHeartO\ 获奖情况}
\datedline{\textit{第一名}, xxx 比赛}{2013 年6 月}

\section{\faInfo\ 其他}
% increase linespacing [parsep=0.5ex]
\begin{itemize}[parsep=0.5ex]
  % \item 技术博客: https://mooonside.github.io/
  \item 语言: 英语 - 熟练(CET-6 547)
\end{itemize}

%% Reference
%\newpage
%\bibliographystyle{IEEETran}
%\bibliography{mycite}
\end{document}
