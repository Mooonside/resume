% !TeX spellcheck = en_US
% !TEX TS-program = xelatex
% !TEX encoding = UTF-8 Unicode
% !Mode:: "TeX:UTF-8"

\documentclass{resume}
\usepackage{zh_CN-Adobefonts_external} % Simplified Chinese Support using external fonts (./fonts/zh_CN-Adobe/)
% \usepackage{NotoSansSC_external}
% \usepackage{NotoSerifCJKsc_external}
% \usepackage{zh_CN-Adobefonts_internal} % Simplified Chinese Support using system fonts
\usepackage{linespacing_fix} % disable extra space before next section
\usepackage{cite}

\begin{document}
	\pagenumbering{gobble} % suppress displaying page number
	
	\name{陈怡峰}
	
	\basicInfo{
		\email{yifengchen@zju.edu.cn} \textperiodcentered\ 
		\phone{(+86) 138-1836-9294} \textperiodcentered\ 
		%\linkedin[yifeng19]{https://www.linkedin.com/in/yifeng19/} \textperiodcentered\ 
		\github[Mooonside]{https://www.github.com/Mooonside/}
	}
	
	\section{\faGraduationCap\  教育背景}
	\datedsubsection{\textbf{浙江大学},杭州,浙江}{2018 -- 至今}
	\textit{在读硕士研究生}\ 计算机科学与技术\ 预计 2021 年 3 月毕业\\
	\textit{导师}\ \href{https://person.zju.edu.cn/xilics}{李玺} 
	\datedsubsection{\textbf{上海交通大学},上海}{2014 -- 2018}
	\textit{学士}\ 计算机科学与技术\\
	\textit{学积分排名}\ 31 / 150
	
	%\section{\faUsers\ 实习/项目经历}
	\section{\faUsers\ 项目经历}
	\datedsubsection{\textbf{带有双向连接和遮挡处理的全景分割模型}}{2019年6月 -- 2019年11月}
	\role{} {CVPR 2020在审中}
	\begin{itemize}
		\item 目的: 全景分割任务包含语义分割任务和实例分割任务。为了利用这两个任务间的互补性,需要在它们间建立联系。另外,对于实例间的遮挡问题,需要一种算法来更好地推理遮挡关系。
		\item 方法: 针对两个任务,设计了双向的特征连接。特别地,对于实例分割到语义分割的连接,我们提出了一个可微分操作RoIInlay,使得实例中丢失的全局空间信息可以被恢复。 对于遮挡问题,我们提出了一个后处理算法,通过比较遮挡区域与候选目标在外观上的相似度来推理遮挡。
		\item 效果: 以通用指标PQ来衡量,使用基于ResNet50的网络,该模型在COCO验证集上达到了state-of-the-art~(43.0\%)。
	\end{itemize}
	
	\datedsubsection{\textbf{降低上下文耦合的语义分割模型}}{2019年3月 -- 2019年5月}
	\begin{itemize}
		\item 目的: 降低模型对上下文环境的依赖从而提升其泛化能力。
		\item 方法: 通过随机擦除来构造样本对,要求模型在丢失部分上下文的情况下不仅能够正确地预测类别,对于未擦除区域特征的提取也需要有一致性,从而降低模型提取特征时对上下文环境的依赖。
		\item 效果: 以通用指标mIoU来衡量,在原始数据集(CityScapes)上精度维持相当的情况下,模型的泛化能力增强了。对测试样例进行风格迁移后,模型的性能较原来提升了1.5\%。在一个从未见过的数据集(Apollo DataSet)上,模型的性能较原来有约2\%的提升,个别类别的提升超过 10\%。
	\end{itemize}

	\datedsubsection{\textbf{Chat Assistant}}{2017年5月5日 -- 2017年5月7日}
	%2017年黑客马拉松青年创客奖
	\role{上海交通大学Hackathon 2017} {负责服务器端开发}
	\begin{itemize}
		\item{目的: 在聊天过程中,我们经常遇到一些不熟悉的领域和词汇。这一现象在有年龄代差的人群间更加显著。为了简化切出聊天、搜索、返回聊天的过程,我们设计了这一聊天助手~(Chat Assistant)。}
		\item 方法: 基于IBM的Watson API,对用户可能感兴趣的名词进行抽取,预先检索并返回相关信息。
	\end{itemize}
	
	\section{\faCogs\ IT 技能}
	% increase linespacing [parsep=0.5ex]
	\begin{itemize}[parsep=0.5ex]
		\item 编程语言: 熟悉Python,CUDA,C++等编程语言
		\item 深度学习框架: 熟悉PyTorch,MXNet,了解TensorFlow
	\end{itemize}
	
	\section{\faHeartO\ 获奖情况}
	\datedline{\textit{电院B级奖学金~(授予前10\%)},上海交通大学}{2015年}
	\datedline{\textit{IBM青年创客奖},上海交通大学Hackathon}{2017年}
	
	\section{\faInfo\ 其他}
	% increase linespacing [parsep=0.5ex]
	\begin{itemize}[parsep=0.5ex]
		% \item 技术博客: https://mooonside.github.io/
		\item 语言: 英语 - 熟练(CET-6 577)
	\end{itemize}
	
	%% Reference
	%\newpage
	%\bibliographystyle{IEEETran}
	%\bibliography{mycite}
\end{document}
